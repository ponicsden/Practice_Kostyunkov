\documentclass[a4paper,12pt]{article} % тип документа

% report, book
\usepackage[left=2cm,right=2cm,top=2cm,bottom=2cm,bindingoffset=0cm]{geometry}
% Рисунки
\usepackage{graphicx}
\usepackage{wrapfig}
\graphicspath{{logo/}}
\usepackage{hyperref}
\usepackage[rgb]{xcolor}
\hypersetup{				% Гиперссылки
    colorlinks=true,       	% false: ссылки в рамках
	urlcolor=blue          % на URL
	}
%  Русский язык

\usepackage[T2A]{fontenc}			% кодировка
\usepackage[utf8]{inputenc}			% кодировка исходного текста
\usepackage[english,russian]{babel}	% локализация и переносы


% Математика
\usepackage{amsmath,amsfonts,amssymb,amsthm,mathtools} 


\usepackage{wasysym}

%Заговолок
\author{Костюньков Данила Сергеевич}
\title{Практика в \LaTeX{}}
\date{\today}

\begin{document}
\setlength{\leftskip}{2mm}
\setlength{\rightskip}{2mm}
\newpage
\begin{center}
Министерство науки и высшего образования\\
Федеральное государственное бюджетное образовательное\\
учреждение высшего образования\\
“Московский государственный технический университет\\
имени Н.Э. Баумана\\
(национальный исследовательский университет)”\\
(МГТУ им. Н.Э. Баумана)\\
\hrulefill\\
\end{center}
\begin{center}
\includegraphics[width=5cm]{logo}
\end{center}
\begin{center}
Факультет “Фундаментальные науки”\\
Кафедра “Высшая математика”\\
~\\
~\\
\textbf{\Huge Отчёт\\
по учебной практике\\
за 1 семестр 2020—2021 гг.}\\
~\\
~\\
Руководитель практики, ст. преп. кафедры ФН1\\ $\underset{\text{подпись}}{\underline{\hspace{0.2\textwidth}}}$ Кравченко О.В.\\
 
студент группы ФН1–11 $\underset{\text{подпись}}{\underline{\hspace{0.2\textwidth}}}$ Костюньков Д.С.\\
\vspace{5cm}
\begin{center}
Москва,\\
2020
\end{center}


\end{center}


\newpage	
\tableofcontents
\newpage
\section{Цели и задачи практики} 
\subsection{Цели}
--- развитие компетенций, способствующих успешному освоению материала бакалавриата и необходимых в будущей профессиональной деятельности.

\subsection{Задачи}
\begin{itemize}


\item Знакомство с программными средствами, необходимыми в будущей профессиональной деятельности.
\item Развитие умения поиска необходимой информации в специальной литературе и других источниках.
\item Развитие навыков составления отчётов и презентации результатов.
\end{itemize}
\subsection{Индивидуальное задание} 
\begin{itemize}
\item Изучить способы отображения математической информации в системе вёртски \LaTeX.
\item Изучить возможности системы контроля версий \textsf{Git}.
\item Научиться верстать математические тексты, содержащие формулы и графики в системе \LaTeX.
Для этого, выполнить установку свободно распространяемого дистрибутива \textsf{TeXLive} и оболочки \textsf{TeXStudio}.
\item Оформить в системе \LaTeX типовые расчёты по курсе математического анализа согласно своему варианту.
\item Создать аккаунт на онлайн ресурсе \textsf{GitHub} и загрузить исходные \textsf{tex}--файлы 
и результат компиляции в формате \textsf{pdf}.

\end{itemize} 

\newpage
\section{Отчёт}
Актуальность темы продиктована необходимостью владеть системой вёрстки \LaTeX и средой вёрстки \textsf{TeXStudio} для
отображения текста, формул и графиков. Полученные в ходе практики навыки могут быть применены при написании
курсовых проектов и дипломной работы, а также в дальнейшей профессиональной деятельности.

Ситема вёрстки \LaTeX содержит большое количество инструментов (пакетов), упрощающих отображение информации в различных 
сферах инженерной и научной деятельности. 

\newpage
\section{Индивидуальное задание}
%\subsection{Элементарные функции и их графики.}
%\input{src / part1. tex}

%=================================================================================================================================
\subsection{Пределы и непрерывность.}
\begin{center}

\textbf{Задача № 1.}
\end{center}\\
\textbf{Условие.} Дана последовательность $\left\lbrace a_n \right\rbrace=\frac{4n^2+1}{3n^2+2}$ и число $c=\frac{4}{3}$.Доказать, что \[ \lim \limits_{n \to \infty} \left\lbrace {a_n} \right\rbrace =c,\]\\
а именно, для каждого сколь угодно малого числа $\varepsilon$ >0 найти наименьшее натуральное
число $N = N(\varepsilon )$ такое, что |$ a_n $ $-$ $c$ | < $\varepsilon$ для всех номеров n > $N(\varepsilon)$. Заполнить таблицу\\
\begin{center}

\begin{tabular}{|c|c|c|c|}
\hline 
$\varepsilon$ & 0,1 & 0,01 & 0,001 \\ 
\hline 
$N(\varepsilon)$ & {} & {} & {} \\ 
\hline 
\end{tabular} 
\end{center}\\
\textbf {Решение.} Рассмотрим неравенство $ a_n $ $-$ $c$ < $\varepsilon$ ,$\forall \varepsilon$>0, учитывая выражение для $ a_n $ и значение $c$ из условия варианта, получим\\
\begin{center}
\[\left|\frac{4n^2+1}{3n^2+2}-\frac{4}{3}\right|<\varepsilon\]
\end{center}\\
Неравенство запишем в виде двойного неравентсва и приведём выражение под знаком модуля к общему знаменателю, получим\\
\begin{center}
\[-\varepsilon<-\frac{5}{3(3n^2+3)}<\varepsilon\]\\
\end{center}
Заметим, что правое неравенство выполнено для любого номера $n \in N$ поэтому, будем
рассматривать левое неравенство\\
\begin{center}
\[-\frac{5}{3(3n^2+3)}>\varepsilon\]\\
\end{center}\\
Выполнив цепочку преобразований, перепишем неравенство относительно $n^2$, и учитывая,что $n \in N$, получим \\
\begin{center}
\[-\frac{5}{3(3n^2+3)}>\varepsilon\]\\
\[(3n^2+3)<-\frac{5}{3\varepsilon}\]\\
\[n^2<\frac{1}{3}(-\frac{5}{3\varepsilon}\]\\
\[n>\frac{1}{3}\sqrt{\frac{5+9\varepsilon}{\varepsilon}}\]\\
\[N(\varepsilon)=\left[\frac{1}{3}\sqrt{\frac{5+9\varepsilon}{\varepsilon}}\right]\]\\ 
\end{center}\\
где [ ] — целая часть числа. Заполним таблицу:\\
\begin{center}

\begin{tabular}{|c|c|c|c|}
\hline 
\varepsilon & 0,1 & 0,01 & 0,001 \\ 
\hline 
\[N(\varepsilon)\] & 2 & 7 & 23 \\ 
\hline 
\end{tabular} 
\end{center}\\
\textbf{Проверка:}\\
\begin{center}
\begin{aligned}
& |$ a_3 $ $-$ $c$ |=$\frac{-5}{87}<0,1$\\
& |$ a_8 $ $-$ $c$ |=$\frac{-5}{582}<0,01$\\
& |$ a_24 $ $-$ $c$ |=$\frac{-5}{5190}<0,001$\\
\end{aligned}
\end{center}\\
\begin{center}

\textbf{Задача № 2.}

\end{center}\\
\textbf{Условие.} Вычислить пределы функций\\
\[(a):  \lim \limits_{x \to 4} \frac{x^2-3x-4}{x^3-6x^2+32}  ,\]\\
\[(b):  \lim \limits_{x \to 0} \frac{\sqrt{1-x}-\sqrt{1+x}}{x}  ,\]\\
\textbf {Решение.}\\
$(a):$\\
\[\lim \limits_{x \to 4} \frac{x^2-3x-4}{x^3-6x^2+32}=\lim \limits_{x \to 4} \frac{(x-4)\cdot(x+1)}{(x-4)^2 \cdot(x+2)} =\infty \]\\
$(b):$\\
\[\lim \limits_{x \to 0} \frac{\sqrt{1-x}-\sqrt{1+x}}{x}=\lim \limits_{x \to 0} \frac{\frac{\sqrt{1-x}}{x}-\frac{\sqrt{1+x}}{x}}{1}=\lim \limits_{x \to 0} \frac{-1}{1}=-1\]\\
\begin{center}

\textbf{Задача № 3.}

\end{center}\\
\textbf{Условие.}\\
$(a):$ Показать, что данные функции $f(x)$ и $g(x)$ являются бесконечно малыми или бесконечно большими при указанном стремлении аргумента.\\
$(b):$ Для каждой функции $f(x)$ и $g(x)$ записать главную часть (эквивалентную ей функцию) вида $C(x- x_0)^\alpha$ при $x \longrightarrow x_0$ или $Cx^\alpha$ при $x \longrightarrow \infty $, указать их порядки малости(роста).\\
$(c):$ Сравнить функции $f(x)$ и $g(x)$ при указанном стремлении.\\
\begin{center}
\begin{tabular}{|c|c|c|}
\hline 
№Варианта & функции $f(x)$ и $g(x)$  & стремление \\ 
9 & $f(x)=2x^3-5x^2+1,g(x)=\frac{1}{1-\cos\frac{1}{\sqrt{x-1}}}$ & x\longrightarrow\infty \\
\hline 
\end{tabular} 
\end{center}\\
\textbf {Решение.}\\
$(a):$ Покажем, что $f(x)$ и $g(x)$ бесконечно большие функции,\\
\begin{center}
\[\lim \limits_{x \to \infty} f(x)= \lim \limits_{x \to \infty} 2x^3-5x^2+1;\lim \limits_{x} \to \infty}(1)^{-5\cdot x^2}=1 \longrightarrow \lim \limits_{x} \to \infty} 2x^3-5x^2+1=\infty;\]\\
\[\lim \limits_{x \to \infty} g(x)= \lim \limits_{x \to \infty}\frac{1}{1-\cos\frac{1}{\sqrt{x-1}}}=\infty;\]\\ 
\end{center}
$(b):$ \[f(x)\thicksim 2x^3 ,\text{при } x\longrightarrow\infty;\]\\
\[g(x)\thicksim x ,\text{при } x\longrightarrow\infty;\]\\
$(c):$ Для сравнения функций $f(x)$ и $g(x)$ рассмотрим предел их отношения при указанном
стремлении\\
\[\lim \limits_{x \to \infty} \frac{f(x)}{g(x)}\]\\
Применим эквивалентности, определенные в пункте $(b):$, получим\\
\[\lim \limits_{x \to \infty} \frac{f(x)}{g(x)}=\lim \limits_{x \to \infty}\ \frac{2x^3}{x}=\lim \limits_{x \to \infty}\ 2x^2=\infty;\]\\
Отсюда, $f(x)$ есть бесконечно большая функция более высокого порядка роста, чем $g(x)$.



%=================================================================================================================================
%\subsection{Приложения дифференциального исчисления.}
%\input{src / part3. tex}

\newpage
\addcontentsline{toc}{section}{Список литературы}
\begin{thebibliography}{99}
\bibitem{book01} {\color{red}\href{https://ru.m.wikibooks.org/wiki/Математические_формулы_в_LaTeX}{wikipedia}.}
\bibitem{book02} {\color{red}\href{https://www.mccme.ru/free-books/llang/newllang.pdf}{Львовский С.М. Набор и вёрстка в системе \LaTeX, 2003 c}.}
\bibitem{book03} {\color{red}\href{https://www.youtube.com/watch?v=I5LM0HU2Ugo}{Курс по \LaTeX на Ютуб}.}
\end{thebibliography}

\end{document}